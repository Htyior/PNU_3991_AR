\documentclass{article}
\usepackage{multicol}
\usepackage{xcolor}
\usepackage{graphicx}
\linespread{1.35}
\usepackage{amsmath}
\usepackage{color}
\usepackage{tikz}
\usetikzlibrary{arrows,automata}

\begin{document}

\begin{flushright}
 \texttt{Finite Automata} \hspace*{0.1cm}\textbf{$|$} \hspace*{0.1cm} \textbf{101}\hspace*{0.1cm}
\end{flushright}
\vspace*{1.5cm}

\hspace*{0.5cm} The transitional diagram is\\

\vspace*{0.7cm}
\begin{center}
\section{picture}
\includegraphics[width=10cm,height=3cm]{101-1.png}
\end{center}

12. Convert the following NFA with $\varepsilon$ move to an equivalent DFA.\\

\vspace*{0.3cm}
\begin{center}
\section{picture}
\includegraphics[width=6cm,height=3cm]{101-2.png}
\end{center}

\vspace*{0.3cm}
\textbf{Solution:} In the given automata, there are two $\varepsilon$ moves: from $q_0$ to $q_1$ and from $q_1$ to $q_2$. If we want
to remove the first $\varepsilon$ move, from $q_0$ to $q_1$, then we have to find all the edges starting from $q_1$.The
edges are $q_1$ to $q_1$ for input a and $q_1$ to $q_2$ for input $\varepsilon$.\\

\hspace*{0.5cm} Duplicate all these transitions starting from the state $q_0$, keeping the edge label the same. $q_0$ is
the initial state, and so make $q_1$ also an initial state. $q_1$ is the final state, and so make $q_0$ as the final
state The modified transitional diagram will be\\

\vspace*{0.3cm}
\begin{center}
\section{picture}
\includegraphics[width=6cm,height=3cm]{101-3.png}
\end{center}

\vspace*{0.3cm}
Again, there are two $\varepsilon$ moves, from $q_0$ to $q_2$ and from $q_1$ to $q_2$. If we want to remove the null transition
from $q_0$ to $q_2$, we have to find all the edges starting from $q_2$.The edges are $q_2$ to $q_0$ for input b.
Duplicate this transition starting from the state $q_0$, keeping the edge label the same. The modified
transitional diagram will be as follows. (As in $q_0$ there is a loop with label b, we need not make
another loop with the same label).\\

\vspace*{0.2cm}
\begin{center}
\section{picture}
\includegraphics[width=6cm,height=3cm]{101-4.png}
\end{center}

\newpage

\begin{flushleft}
    \textbf{102}\hspace*{0.1cm} \textbf{$|$} \hspace*{0.1cm} \texttt{Introduction to Automata Theory, Formal Languages and Computation}
  \end{flushleft}
  \vspace*{1cm}

  If we want to remove the null transition from $q_1$ to $q_2$, we have to find all the edges starting from
$q_2$.The edges are $q_2$ to $q_0$ for input b. Duplicate this transition starting from the state $q_1$, keeping
the edge label the same. $q_1$ is the initial state, and so make $q_2$ also an initial state. The modified
transitional diagram will be\\

\vspace*{0.3cm}
\begin{center}
\section{picture}
\includegraphics[width=6cm,height=3cm]{102.png}
\end{center}

\vspace*{0.3cm}
13. Convert the following Moore machine to an equivalent Mealy machine by the tabular format.\\

\vspace*{0.2cm}
\begin{center}
\begin{tabular}{cccc}
 \hline

 \hline

 \hline

 \hline
 & \multicolumn{2}{c}{$Next State$}\\
 \cline{2-3}
 $State$ &  $I/P=0$ & $I/P=1$  &  $O/P$\\
\hline
$\rightarrow q_0$  &    $q_2$   &  $q_1$   &  $0$ \\
$q_1$              &    $q_0$   &  $q_3$   &  $1$ \\
$q_2$              &    $q_3$   &  $q_4$   &  $1$ \\
$q_3$              &    $q_4$   &  $q_1$   &  $0$ \\
$q_4$              &    $q_4$   &  $q_2$   &  $1$ \\
 \hline

 \hline

 \hline

 \hline
\end{tabular}
\end{center}

\vspace*{0.3cm}
\textbf{Solution:} The equivalent Mealy machine is\\

\vspace*{0.2cm}
\begin{center}
\begin{tabular}{ccccc}
 \hline

 \hline

 \hline

 \hline
 &  \multicolumn{2}{c}{$I/P = 0$ } &  \multicolumn{2}{c}{$I/P = 1$}  \\
  \cline{2-3}                         \cline{4-5}
 $Present State$ &   $Next State$  & $O/P$ &  $Next State$  & $O/P$\\
\hline
$\rightarrow q_0$  &  $q_2$  &  $1$  &  $q_1$  & $1$ \\
$q_1$             &  $q_0$  &  $0$  &  $q_3$  & $0$ \\
$q_2$             &  $q_3$  &  $0$  &  $q_4$  & $1$ \\
$q_3$             &  $q_4$  &  $1$  &  $q_1$  & $1$ \\
$q_4$             &  $q_4$  &  $1$  &  $q_2$  & $1$ \\
 \hline

 \hline

 \hline

 \hline
\end{tabular}
\end{center}

\vspace*{0.3cm}
14. Convert the following Moore machine to an equivalent Mealy machine by the tabular format.\\

\vspace*{0.2cm}
\begin{center}
\begin{tabular}{cccc}
 \hline

 \hline

 \hline

 \hline
 & \multicolumn{2}{c}{$Next State$ }\\
 \cline{2-3}
 $State$ &  $I/P=0$ & $I/P=1$  &  $O/P$\\
\hline
$\rightarrow q_0$  &    $q_1$   &  $q_3$   &  $1$ \\
$q_1$              &    $q_0$   &  $q_2$   &  $0$ \\
$q_2$              &    $q_4$   &  $q_1$   &  $0$ \\
$q_3$              &    $q_4$   &  $q_3$   &  $1$ \\
$q_4$              &    $q_3$   &  $q_1$   &  $0$ \\
 \hline

 \hline

 \hline

 \hline
\end{tabular}
\end{center}

\newpage

\begin{flushright}
 \texttt{Finite Automata} \hspace*{0.1cm}\textbf{$|$} \hspace*{0.1cm} \textbf{103}\hspace*{0.1cm}
\end{flushright}
\vspace*{0.5cm}

\textbf{Solution:} The equivalent Mealy machine is\\

\begin{center}
\begin{tabular}{ccccc}
 \hline

 \hline

 \hline

 \hline
 &  \multicolumn{2}{c}{$I/P = 0$ } &  \multicolumn{2}{c}{$I/P = 1$}  \\
  \cline{2-3}                         \cline{4-5}
 $Present State$ &   $Next State$  & $O/P$ &  $Next State$  & $O/P$\\
\hline
$\rightarrow q_0$  &  $q_1$  &  $0$  &  $q_3$  & $1$ \\
$q_1$             &  $q_0$  &  $1$  &  $q_2$  & $0$ \\
$q_2$             &  $q_4$  &  $0$  &  $q_1$  & $0$ \\
$q_3$             &  $q_4$  &  $1$  &  $q_1$  & $1$ \\
$q_4$             &  $q_3$  &  $1$  &  $q_1$  & $0$ \\
 \hline

 \hline

 \hline

 \hline
\end{tabular}
\end{center}

\vspace*{0.3cm}
15. Convert the following Mealy machine to an equivalent Moore machine by the tabular format.

\begin{center}
\begin{tabular}{ccccc}
 \hline

 \hline

 \hline

 \hline
 &  \multicolumn{2}{c}{$I/P = 0$ } &  \multicolumn{2}{c}{$I/P = 1$}  \\
  \cline{2-3}                         \cline{4-5}
 $Present State$ &   $Next State$  & $O/P$ &  $Next State$  & $O/P$\\
\hline
$\rightarrow q_0$  &  $q_1$  &  $1$  &  $q_2$  & $1$ \\
$q_1$             &  $q_3$  &  $0$  &  $q_0$  & $1$ \\
$q_2$             &  $q_4$  &  $0$  &  $q_3$  & $1$ \\
$q_3$             &  $q_1$  &  $0$  &  $q_4$  & $0$ \\
$q_4$             &  $q_2$  &  $1$  &  $q_4$  & $0$ \\
 \hline

 \hline

 \hline

 \hline
\end{tabular}
\end{center}

\vspace*{0.3cm}

\textbf{Solution:} In the next state column of the given Mealy machine, the output differs for $q_1$ and $q_3$ as
the next states. So, the states will be divided as $q_10$, $q_11$ and $q_30$, $q_31$, respectively. After dividing the
states, the modified Mealy machine becomes\\

\begin{center}
\begin{tabular}{ccccc}
 \hline

 \hline

 \hline

 \hline
 &  \multicolumn{2}{c}{$I/P = 0$ } &  \multicolumn{2}{c}{$I/P = 1$}  \\
  \cline{2-3}                         \cline{4-5}
 $Present State$ &   $Next State$  & $O/P$ &  $Next State$  & $O/P$\\
\hline
$\rightarrow q_0$  &  $q_11$  &  $1$  &  $q_2$  & $1$ \\
$q_10$             &  $q_30$  &  $0$  &  $q_0$  & $1$ \\
$q_11$             &  $q_30$  &  $0$  &  $q_0$  & $1$ \\
$q_2$             &  $q_4$    &  $0$  &  $q_4$  & $1$ \\
$q_30$            &  $q_10$   &  $0$  &  $q_31$ & $0$ \\
$q_31$            &  $q_10$   &  $0$  &  $q_4$  & $0$ \\
$q_4$             &  $q_2$    &  $1$  &  $q_4$  & $0$ \\
 \hline

 \hline

 \hline

 \hline
\end{tabular}
\end{center}

\vspace*{0.3cm}
The converted Moore machine is\\

\vspace*{0.3cm}

\begin{center}
\begin{tabular}{cccc}
 \hline

 \hline

 \hline

 \hline
 & \multicolumn{2}{c}{$Next State$ }\\
 \cline{2-3}
 $State$ &  $I/P=0$ & $I/P=1$  &  $O/P$\\
\hline
$\rightarrow q_0$   &    $q_11$   &  $q_2$   &  $1$ \\
$q_10$              &    $q_30$   &  $q_0$   &  $0$ \\

 \hline

\end{tabular}
\end{center}

\newpage

\begin{flushleft}
    \textbf{104}\hspace*{0.1cm} \textbf{$|$} \hspace*{0.1cm} \texttt{Introduction to Automata Theory, Formal Languages and Computation}
  \end{flushleft}
  \vspace*{1cm}

  \begin{center}
\begin{tabular}{cccc}
 \hline

 \hline

 \hline

 \hline
 & \multicolumn{2}{c}{$Next State$ }\\
 \cline{2-3}
 $State$ &  $I/P=0$ & $I/P=1$  &  $O/P$\\
\hline
$q_11$   &    $q_30$   &  $q_0$    &  $1$ \\
$q_2$    &    $q_4$    &  $q_31$   &  $1$ \\
$q_30$   &    $q_10$   &  $q_4$    &  $0$ \\
$q_31$   &    $q_10$   &  $q_4$    &  $1$ \\
$q_4$    &    $q_2$    &  $q_4$    &  $0$ \\
 \hline

 \hline

 \hline

 \hline
\end{tabular}
\end{center}

\vspace*{0.3cm}
To get rid of the problem of occurrence of a null string, we need to include another state, $q_a$, with
the same transactions as that of $q_0$ but with output 0.\\

\hspace*{0.5cm} The modified final Moore machine equivalent to the given Mealy machine is\\

  \begin{center}
\begin{tabular}{cccc}
 \hline

 \hline

 \hline

 \hline
 & \multicolumn{2}{c}{$Next State$ }\\
 \cline{2-3}
 $State$ &  $I/P=0$ & $I/P=1$  &  $O/P$\\
\hline
$q_a$    &    $q_11$   &  $q_2$    &  $0$ \\
$q_0$    &    $q_11$   &  $q_2$    &  $1$ \\
$q_10$   &    $q_30$   &  $q_0$    &  $0$ \\
$q_11$   &    $q_30$   &  $q_0$    &  $1$ \\
$q_2$    &    $q_4$    &  $q_31$   &  $1$ \\
$q_30$   &    $q_10$   &  $q_4$    &  $0$ \\
$q_31$   &    $q_10$   &  $q_4$    &  $1$ \\
$q_4$    &    $q_2$    &  $q_4$    &  $0$ \\
 \hline

 \hline

 \hline

 \hline
\end{tabular}
\end{center}

\vspace*{0.3cm}
16. Convert the following Mealy machine to an equivalent Moore machine by the tabular format.\\

\begin{center}
\begin{tabular}{ccccc}
 \hline

 \hline

 \hline

 \hline
 &  \multicolumn{2}{c}{$I/P = 0$ } &  \multicolumn{2}{c}{$I/P = 1$}  \\
  \cline{2-3}                         \cline{4-5}
 $Present State$ &   $Next State$  & $O/P$ &  $Next State$  & $O/P$\\
\hline
$\rightarrow q_0$  &  $q_0$   &  $1$  &  $q_1$  & $0$ \\
$q_1$             &  $q_3$    &  $1$  &  $q_3$  & $1$ \\
$q_2$             &  $q_1$    &  $1$  &  $q_2$  & $1$ \\
$q_3$             &  $q_2$    &  $0$  &  $q_0$  & $1$ \\

 \hline

 \hline

 \hline

 \hline
\end{tabular}
\end{center}

\vspace*{0.3cm}

\textbf{Solutuion:} In the next state column of the given Mealy machine, the output differs for $q_1$ and $q_2$
as the next states. So, the states will be divided as $q_10$, $q_11$ and $q_20$, $q_21$, respectively. After dividing
the states, the modified Mealy machine will be\\

\vspace*{0.3cm}
\begin{center}
\begin{tabular}{ccccc}
 \hline

 \hline

 \hline

 \hline
 &  \multicolumn{2}{c}{$I/P = 0$ } &  \multicolumn{2}{c}{$I/P = 1$}  \\
  \cline{2-3}                         \cline{4-5}
 $Present State$ &   $Next State$  & $O/P$ &  $Next State$  & $O/P$\\
\hline
$\rightarrow q_0$  &  $q_0$    &  $1$  &  $q_10$  & $0$ \\
$q_10$             &  $q_3$    &  $1$  &  $q_3$   & $1$ \\
$q_11$             &  $q_3$    &  $1$  &  $q_3$   & $1$ \\
 \hline
\end{tabular}
\end{center}


\end{document} 

